% Resumo em língua vernácula
\begin{center}
	{\Large{\textbf{Especificação e Desenvolvimento do Sistema de Materiais (SiMate) para o Instituto Metrópole Digital}}}
\end{center}

\vspace{1cm}

\begin{flushright}
	Autor: Wendell Pamplona Barreto\\
	Orientador(a): Prof. Dr. Marcel Oliveira
\end{flushright}

\vspace{1cm}

\begin{center}
	\Large{\textsc{\textbf{Resumo}}}
\end{center}

\noindent A maneira como as organizações e instituições dependem dos sistemas de informações pra alavancar seus desempenhos e se sobressair no mercado é notável. Diante dos estudos da tecnologia da informação, soluções são desenvolvidas com foco em optimizar e aprimorar os processos. Perante essas necessidades, este trabalho propõe uma solução de software que busca optimizar o processo de criação de materiais no setor de produção de materiais do Instituto Metrópole Digital, unidade suplementar da Universidade Federal do Rio Grande do Norte. Através dessa proposta, o estudo de caso será realizado para levantar os requisitos necessários para o desenvolvimento, assim como validá-los com os envolvidos, garantindo assim que o modelo de software proposto atenda a demanda dos usuários. 

\noindent\textit{Palavras-chave}: sistema de informação, desenvolvimento, processos, materiais.