\chapter{Conclusões}

\section{Considerações e Conclusões}

O \textbf{Gema} é uma solução de software desenvolvida com o intuito de potencializar as atividades que permeiam a gestão de produção de materiais multimídia. O uso dela reduz os gastos com alternativas de controle manual e aumenta a eficácia e visibilidade no cumprimento de cada etapa do processo. Tarefas primordiais como as de notificação de prazos são removidas da responsabilidade da equipe de gerência e determinam comportamentos de sistema, sendo esses executados com precisão diária a garantir o recebimento de cada alerta e aviso por parte dos usuários.

No que se diz respeito à arquitetura usada, o \textbf{Gema} se baseia no Framework de desenvolvimento do IMD, dessa maneira, o trabalho aqui realizado serve de documentação em uma linguagem comum para os desenvolvedores do setor de desenvolvimento do instituto. Os componentes produzidos na aplicação possuem propriedades para serem facilmente estendidos e reutilizados, buscando trazer importantes benefícios para o processo de manutenção do sistema.

As aplicações dos conceitos da engenharia de software e de projeto de interface de usuário nesse estudo foram importantes para que o desenvolvimento fosse realizado acentuando os aspectos de usabilidade e garantindo a capacidade de expansão do sistema. 

O especificação do \textbf{Gema} trouxe benefícios tanto para o desenvolvimento da aplicação  quanto prospecta a publicação base para uso e manutenção. Ao término desse trabalho, o sistema atua no Setor de Produção Multimídia cumprindo suas expectativas e abrindo a visão dos usuários para a constante evolução do processo produtivo. 

\section{Trabalhos Futuros}

Utilizando o \textbf{Gema}, é possível executar fluxos genéricos dos materiais de caráter didático que compõem as ofertas de disciplina. A Oferta representa o modelo base da necessidade inicial do setor, mas não deve ser o único a compor o sistema. Dessa maneira, o primeiro dos trabalhos futuros se justifica na extensão do sistema com a implementação de modelos que determinam a produção de materiais para eventos, cursos e demais situações que fazem ou farão parte do escopo de produção do instituto.

%Atualmente, o fluxo de produção multimídia que acontece no sistema começa após a elaboração da primeira versão do material, isto ocorre pois o sistema não suporta o gerenciamento de conteúdo multimídia como criação e edição de textos, imagens, áudio e vídeo. Dado que ainda não existe abertura para o suporte eficaz à gerência de imagens, áudio e vídeo na WEB, mas alternativas sólidas para a edição de documentos de texto já são passivas de implementação, uma das funcionalidades pretendidas para o \textbf{Gema} é a de gestão de conteúdo de documentos com ferramentas de configuração de estilo, edição multiusuário em tempo real, interação através de comentários, controle de versão de mudanças e exportação do conteúdo para as extensões comumente usadas.

Um das maiores preocupações no desenvolvimento da ferramenta foi a de evidenciar as necessidades do usuário. A visualização das informações pertinentes a cada etapa do fluxo são realçadas para auxiliar o usuário a cumprir o processo da maneira mais eficaz possível. Mantendo essa abordagem, pretendemos implementar mecanismos estatísticos com métricas e produção de relatórios para que se possa analisar o sistema pós execução dos fluxos. Esses mecanismos servirão para que os gerentes possam visualizar o desenvolver dos autores e equipes em cada nincho de produção.
 
Outra face do sistema que enfatiza a visualização do que está acontecendo no fluxo é a de notificação. Esse mecanismo se mostra sensível ao passo que precisa cumprir duas funções cruciais: a de carregar o que se quer informar e a de ser eficiente, ou seja, ser atrativo para que o usuário busque e entenda a informação passada. Pela importância da efetividade encontrada no cumprimento dessas funções, aperfeiçoá-las é um dos objetivos planejados nos trabalhos futuros. 
 
Finalmente, o sistema \textbf{Gema} busca cada vez mais a integralização na plataforma de aplicações do Instituto Metrópole Digital, sendo assim, planejamos realizar essa ponte de comunicação entre os sistemas para que, fornecendo e colhendo dados, se possa potencializar os serviços oferecidos.
 
 
 
 
 